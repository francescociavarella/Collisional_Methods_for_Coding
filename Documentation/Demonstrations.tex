\documentclass[a4paper,12pt]{article}

\usepackage[utf8]{inputenc}
\usepackage[T1]{fontenc}
\usepackage{titlesec}
\usepackage{amsmath}
\usepackage{amssymb}
\usepackage{hyperref}
\usepackage{braket}
\usepackage{tikz}
\usepackage{cancel}
\usepackage[top=3cm, bottom=3cm, left=2cm, right=2cm]{geometry}
\usepackage{graphicx}
\usepackage{float}
\usetikzlibrary{arrows.meta, angles, quotes}

\allowdisplaybreaks

\titleformat{\section}
{\normalfont\normalsize\bfseries}
{\thesection}{1em}{}
\titleformat{\subsection}
{\normalfont\small\bfseries}
{\thesubsection}{1em}{}
\titleformat{\subsubsection}
{\normalfont\small\bfseries}
{\thesubsubsection}{1em}{}

\title{
	\vspace{2cm}
	\Huge \textbf{Demontrations} \\[0.5cm]
	\large Collection of notes, formulas and demonstrations in the field of dynamics of open quantum systems and Collisional Methods
	\vspace{5cm}
}

\author{\Large Francesco Ciavarella - University of Padua \\[10pt] \href{mailto:francesco.ciavarella@studenti.unipd.it}{francesco.ciavarella@studenti.unipd.it}}

\begin{document}
	
	\maketitle
	\thispagestyle{empty} 
	\newpage
	
	\tableofcontents
	\newpage
	
	\section{Quantum Jump limit in Collisional Methods}
	\subsection{Evolution Operator and Ancilla Initialization}
	The goal is to reproduce the Lindbald dynamic using a Collsional Model, that gives us a trajectory of evolution of a single state of the density matrix $ \ket{\Psi_k} $ rappresentig a qubit. By repeting the dynamic several times it's possibile to recrate the Lindblad evolution of the density matrix.
	Different unravelling of the $ \rho (t) $ can be achieved by using different Collsional Hamiltonian. First we will reproduce the so called "Quantum Jump" or "Monte Carlo" limit, in which essentially the ancilla states stochastically jumps between the states $ \ket{0_a} $ and $ \ket{1_a} $ with a small probability to move to $ \ket{1} $ related to an effective collision, wich corresponds to an application of a $ \sigma_{z} $ to the system's state. \\
	The associated Hamiltonian acting on the System-Ancilla states is:
	\begin{align}
		&\mathcal{H}_{CM} = \mathcal{H}_{exc} + \mathcal{H}_{collision} \\
		&= \sum_{j=1}^{N} {\frac{\varepsilon_j}{2} \sigma_{z}^{j} \otimes \mathbb{I}^{\otimes N} } + \sum_{\langle j,j' \rangle} {\frac{V_{j,j'}}{2} \left( \sigma_{x}^{j}\sigma_{x}^{j'} \otimes \mathbb{I}^{\otimes N} + \sigma_{y}^{j}\sigma_{y}^{j'} \otimes \mathbb{I}^{\otimes N} \right)} + \sum_{j=1}^{N}{c_j \sigma_{z}^{j} \otimes \sigma_{x}^{a_j}}
	\end{align}
	where $ \sigma_{\alpha}^j = \mathbb{I}^{\otimes j-1} \otimes \sigma_{\alpha} \otimes \mathbb{I}^{\otimes N-j}  $ . \\
	The collision force $ c_j $ is related to the Lindblad phase shift constant by the equation $ c_j = \sqrt{\gamma_j / 4 \Delta t} $. 
	Let's focus on the Interaction Hamiltonian and how it acts on an already enatgled system-ancilla states $ \ket{\Psi} = \ket{\Psi_S} \otimes \ket{0_a} $ \\
	\\
	The evolution operator based on $ \mathcal{H}_{collsion} $ is :
	\begin{equation}
		U_{collsion} = \exp \left(-i c_j \sigma_{z}^{j} \otimes \sigma_{x}^{a_j} \Delta t\right) = \cos \left(c_j \Delta t \right) \mathbb{I}^j \otimes \mathbb{I}^a - i \sin \left(c_j \Delta t \right) \sigma_{z}^{j} \otimes \sigma_{x}^{a_j}
	\end{equation}
	And applied to $ \ket{\Psi} $ gives:
	\begin{align} 
		\ket{\Psi'} &= \cos \left(c_j \Delta t \right) \mathbb{I}^j \otimes \mathbb{I}^a \ket{\Psi} - i \sin \left(c_j \Delta t \right) \sigma_{z}^{j} \otimes \sigma_{x}^{a} \ket{\Psi} \notag \\
		&= \cos \left(c_j \Delta t \right) \ket{\Psi_S} \otimes \ket{0_a} - i \sin \left(c_j \Delta t \right) \sigma_{z}^{j} \ket{\Psi_S} \otimes \sigma_{x}^{a} \ket{0_a} \notag \\
		&= \cos \left(c_j \Delta t \right) \ket{\Psi_S} \otimes \ket{0_a} - i \sin \left(c_j \Delta t \right) \sigma_{z}^{j} \ket{\Psi_S} \otimes \ket{1_a}
	\end{align}
	
	\clearpage
	\newpage
	
	\subsection{Reconstruction of the Density Matrix}
	Now we can rebuild the Density Matrix as :
	\begin{align}
		\rho' &= \ket{\Psi'}\bra{\Psi'} = \cos^{2} \left(c_j \Delta t \right) \rho_S' \otimes \ket{0_a}\bra{0_a} + \sin^{2} \left(c_j \Delta t \right) \sigma_{z}^{j} \rho_S \sigma_{z}^{j} \otimes \ket{1_a}\bra{1_a} \notag \\
		&+ \cos \left(c_j \Delta t \right) \sin \left(c_j \Delta t \right) \rho_S \sigma_{z}^{j} \otimes \ket{0_a} \bra{1_a} + \cos \left(c_j \Delta t \right) \sin \left(c_j \Delta t \right) \sigma_{z}^{j} \rho_S \otimes \ket{1_a} \bra{0_a}
	\end{align}
	
	\begin{flushleft}
		If we now take the partial Trace over our total density matrix we obtain the System Density Matrix:
	\end{flushleft}
	
	\begin{align}
		\rho'_S & = Tr_a(\rho') = \sum_{k=1}^{N_{anc}}{\bra{\phi_k}\rho'\ket{\phi_k}} \notag \\
		&= \cos^{2}{\left(c_{j} \Delta t\right)} \rho_S + \sin^{2}{\left(c_{j} \Delta t\right)} \, \sigma_{z} \rho_S \sigma_{z} 
	\end{align}	
	
	\subsection{Recovery of Lindblad form}
	Before working on Eq.10 we have to point out that so far we have only analyzed the effect of the Collisional Hamiltonian, forgetting about the unitary evolution via the site Hamiltonian, which has the form : 
	
	\begin{equation}
		\mathcal{H}_{exc} = \sum_{j=1}^{N} {\frac{\varepsilon_j}{2} \sigma_{z}^{j} \otimes \mathbb{I}^{\otimes N} } + \sum_{\langle j,j' \rangle} {\frac{V_{j,j'}}{2} \left( \sigma_{x}^{j}\sigma_{x}^{j'} \otimes \mathbb{I}^{\otimes N} + \sigma_{y}^{j}\sigma_{y}^{j'} \otimes \mathbb{I}^{\otimes N} \right)}
	\end{equation}
	
	In this case we already know that applying the Time Evolution Operator $ \mathcal{U}_{exc} = \exp {\left( -i \mathcal{H}_{exc}\Delta t \right)} $ in the first order in $ \Delta t $ approximation, we obtain the Liouville-von Neumann Equation, which can be written as : 
	
	\begin{align}
		\rho_S \left(t + \Delta t\right) = \rho_S (t) - i \left[\mathcal{H}_{exc}, \rho_S \right] \Delta t \\
		\frac{\rho_S \left(t + \Delta t\right) - \rho_S (t)}{\Delta t} = - i \left[\mathcal{H}_{exc}, \rho_S \right]
	\end{align} 
	
	Let's focus now on Eq.9 and its form for $ \lim \Delta t \rightarrow 0  $; using Taylor's expansion for the $ sin $ and $ cos $ we obtain that:
	
	\begin{itemize}
		\item $ \cos^{2}{\left(c_{j} \Delta t \right)} \approx 1 - c_{j}^{2} \Delta t ^{2} $
		\item $ \sin^{2}{\left(c_{j} \Delta t \right)} \approx c_{j}^{2} \Delta t ^{2} $
	\end{itemize} 
	
	\begin{align}
		\rho_S \left(t + \Delta t\right) & = \cos^{2}{\left(c_{j} \Delta t\right)} \rho_S (t) + \sin^{2}{\left(c_{j} \Delta t\right)} \, \sigma_{z} \rho_S (t) \sigma_{z} \notag \\ 
		& \approx \rho_S (t) + c_{j}^{2} \Delta t ^{2} \left( \sigma_{z} \rho_S \sigma_{z} - \rho_S (t) \right)
	\end{align}
	
	\clearpage
	\newpage
	
	The total dynamics over a time step $\Delta t$ can be obtained by composing the unitary evolution generated by $\mathcal{H}_{exc}$ and the collisional map derived above. Using the "Trotter decomposition" to first order in $\Delta t$, we approximate the total evolution operator as the product of the individual propagators:
	
	\begin{equation}
		\mathcal{U}_{tot}(\Delta t) \approx \mathcal{U}_{exc}(\Delta t) \cdot \mathcal{U}_{coll}(\Delta t)
	\end{equation}
	
	Applying this composition sequentially to the density matrix (i.e., applying the Hamiltonian evolution to the state resulting from the collision), we obtain:
	
	\begin{align}
		&\rho_S \left(t + \Delta t\right) = \rho_S (t) - i \left[\mathcal{H}_{exc}, \rho_S \right]\Delta t + \sum_{j=1}^{N}{c_{j}^{2} \Delta t ^{2} \left( \sigma_{z}^{j} \rho_S \sigma_{z}^{j} - \rho_S (t) \right)} \\
		&\frac{\rho_S \left(t + \Delta t\right) - \rho_S (t)}{\Delta t} = - i \left[\mathcal{H}_{exc}, \rho_S \right] + \sum_{j=1}^{N}{c_{j}^{2} \Delta t \left( \sigma_{z}^{j} \rho_S \sigma_{z}^{j} - \rho_S (t) \right)} 
	\end{align}
	
	We have to set $ c_{j}^{2} = \Gamma_j / \Delta t  $ to recover the differential form : 
	
	\begin{equation}
		\lim\limits_{\Delta t \rightarrow 0} \quad \dot{\rho_S} = - i \left[\mathcal{H}_{exc}, \rho_S \right] + \sum_{j=1}^{N}{\Gamma_{j} \left( \sigma_{z}^{j} \rho_S \sigma_{z}^{j} - \rho_S (t) \right)}
	\end{equation}
	
	Eq.20 corresponds to the Lindblad ME setting $ \mathcal{L}_k = \sigma_{z} $ in fact $ \sigma_{z} = \sigma_{z}^{\dagger} $ and $ \sigma_{z} \sigma_{z}^{\dagger} = \mathbb{I} $	
	
	\subsection{Algorithm for Trajectories}
	
	By measuring the Ancilla's state we can deduce if we had an avoided collision $ \ket{0_a} $ or an occurred collision $ \ket{1_a} $, which imples the application to the system of $ \sigma_{z}^{j} \ket{\Psi_S} $ which introduce a flip in the phase of the interacting site, with a probability of $ \left| {\sin \left(c_j \Delta t \right)} \right|^2  $, which contributes to the site dephasing process, helping the excitonic tranport (ENAQT - Enviroment Assisted Quantum Transport). \\
	Measuring repeted time $ s $ the Ancilla states correspond on tracing its states out, so that we could rebuild the density matrix as :
	\begin{equation}
		\rho_S' = \cos^{2}{\left(c_{j} \Delta t\right)} \rho_S + \sin^{2}{\left(c_{j} \Delta t\right)} \, \sigma_{z} \rho_S \sigma_{z} 
	\end{equation}
	
	Now we can proceed in two different ways:
	\begin{itemize}
		\item Evolution of the $ \rho (t) $ with $ U(t) $ and $ U(t)^{\dagger} $; than we trace out the ancilla subsystem with a partial trace over its degree of freedom, in order to get back the average evolution of the system, which gaves a trajectory that reproduces the Lindblad ME. This proves that the Collisional Method gives the same evolution of a Lindblad ME.
		\item Evolution of the only $ \ket{\Psi_{\mathcal{S}}} $ with $ U(t) $ and then the measure on the Ancilla, in order to apply the $ \sigma_z^s $ to the system's state or don't do anything, i.e. apply $ \mathbb{I}^s $. In a Quantistic Computer that can handle the sovrapposition of the ground and excited state of the ancilla, we can just apply the $  U(t) $ build on the $ H_{CM} $ we had already defined.
	\end{itemize}
	
	In a Classical Computer we have to simulate the collision with the Ancilla (so that we never pass in a composed state $ \ket{\Psi_{\mathcal{S}}} \otimes \ket{a} $); this could be done by extracting a random number from a distriubution that replicates the probability given by $ \left| \sin{(c_1 \Delta t)} \right|^2 $, than if the condition in respected it means that an effective collsion has occured and so the ancilla is in the state $ \ket{1_a} $ and so we apply $ \sigma_z^s $ to the system's state; if is not we don't modify the system's state. \\
	In this way we repoduce a stochastic evolution oh the $ \ket{\Psi_{\mathcal{S}}} $, i.e. a single random trajectory; if we generate different trajectories and the mediate over them $ \overline{\ket{\Psi_{\mathcal{S}}} \bra{\Psi_{\mathcal{S}}}} $ we can rebuild the evolution of the density matrix obtained with the Lindblad ME. \\
	The algorithm to create a single trajectories is the following:
	\begin{itemize}
		\item [1.] Apply the unitary evolution due to only $ H_{site} $ and the hopping potential $ V $ for a disctrete time step $ \Delta t $
		\item[2.] Extract a random number between 0 and 1 for every system's site 
		\item[3.] If the extracted number for the single site is lower than $ \left| \sin{(c_i \Delta t)} \right| $ we apply the $ \sigma_z $ operator to that site 
		\item[4.] Record the site popolutation at that time, than restart the algorithm
	\end{itemize}
	
	\clearpage
	\newpage
	
	\section{Diffusive Limit in Collisional Methods}
	
	\subsection{Evolution Operator and Ancilla Initialization}
	
	In this framework, we define the interaction Hamiltonian and the initial state of the ancilla as follows:
	\begin{align} 
		&\mathcal{H}_{CM} = c_{j} \sigma_{z}^{j} \otimes \sigma_{z}^{a_j} \quad \text{and} \quad \rho_{a} = \frac{1}{2} \, \mathbb{I}^{a_j} =\rho_{a} = \frac{1}{2} \begin{pmatrix} 1 & 0 \\ 0 & 1 \end{pmatrix} 
	\end{align}
	
	The definition of the initial state of the Ancilla as a completely mixed state is crucial. It implies that the wave function for the trajectory development will be initialized with probability $p=1/2$ in the state $ \ket{\phi_a} = \ket{0_a} $ and with probability $p=1/2$ in the state $ \ket{\phi_a} = \ket{1_a} $. \\
	
	The Unitary Evolution Operator reads: 
	\begin{equation}
		\mathcal{U} = \exp{\left( -i c_j \Delta t \, \sigma_{z}^{j} \otimes \sigma_{z}^{a_j} \right)}
	\end{equation}
	
	First we expand in the Taylor's Series the exponential which reads:
	\begin{equation}
		\begin{aligned}
			&\exp{\left( -i c_j \Delta t \, \sigma_{z}^{j} \otimes \sigma_{z}^{a_j} \right)} = \, \sum_{n=0}^{\infty}{\frac{\left( -i c_j \Delta t \, \sigma_{z}^{j} \otimes \sigma_{z}^{a_j} \right)^n}{n!} } \\
		\end{aligned}
	\end{equation}
	
	\begin{align*}
		&= 1 - i c_j \Delta t \, \sigma_{z}^{j} \otimes \sigma_{z}^{a_j} - \frac{\left( c_j \Delta t \, \sigma_{z}^{j} \otimes \sigma_{z}^{a_j} \right)^2}{2!} + i \frac{\left( c_j \Delta t \, \sigma_{z}^{j} \otimes \sigma_{z}^{a_j} \right)^3}{3!} + \frac{\left( c_j \Delta t \, \sigma_{z}^{j} \otimes \sigma_{z}^{a_j} \right)^4}{4!} + \cdots  \\
		&=1 - \frac{\left( c_j \Delta t \, \sigma_{z}^{j} \otimes \sigma_{z}^{a_j} \right)^2}{2!} + \cdots  + \frac{\left( -i c_j \Delta t \, \sigma_{z}^{j} \otimes \sigma_{z}^{a_j} \right)^{{2n}}}{2n!}  \\
		&\,- i c_j\Delta t \, \sigma_{z}^{j} \otimes \sigma_{z}^{a_j} + \cdots  + \frac{\left( -i c_j \Delta t \, \sigma_{z}^{j} \otimes \sigma_{z}^{a_j} \right)^{{2n+1}}}{(2n+1)!} \\ 
	\end{align*}
	
	\begin{flushleft}
		We now apply th Pauli Matrices property to calculate the exponential term: 
	\end{flushleft}
	
	\begin{equation}
		\begin{aligned}
			&\left(\sigma_{z}^{j} \otimes \sigma_{z}^{a_j} \right)^{2} = \sigma_{z}^{j} \sigma_{z}^{j} \otimes \sigma_{z}^{a_j} \sigma_{z}^{a_j} = \mathbb{I}^{j} \otimes \mathbb{I}^{a_j} \\
			&\left(\sigma_{z}^{j} \otimes \sigma_{z}^{a_j} \right)^{3} = \sigma_{z}^{j} \otimes \sigma_{z}^{a_j}
		\end{aligned}
	\end{equation}
	
	\clearpage
	\newpage
	
	\begin{flushleft}
		So we can rewrite the summation and make evident the \textit{cos} and \textit{sin} expansion:
	\end{flushleft}
	
	\begin{align}
		\exp{\left( -i c_j \Delta t \, \sigma_{z}^{j} \otimes \sigma_{z}^{a_j} \right)} &=\mathbb{I}^{j} \otimes \mathbb{I}^{a_j} \, \sum_{2n=0}^{\infty}
		{ \frac{\left( c_j \Delta t \right)^{{2n}}}{2n!} } \,  - i \left(\sigma_{z}^{j} \otimes \sigma_{z}^{a_j} \right) \sum_{2n=0}^{\infty}{ \frac{\left( c_j \Delta t  \right)^{{2n+1}}}{(2n+1)!} } \notag \\
		\notag \\	
		&= \cos{\left(c_j \Delta t\right)} \mathbb{I}^{j} \otimes \mathbb{I}^{a_j} \, - \sin{\left(c_j \Delta t \right)} \sigma_{z}^{j} \otimes \sigma_{z}^{a_j}
	\end{align}
	
	\begin{flushleft}
		Now we can apply the evolution operator to the two different initial state which are :
	\end{flushleft}
	
	\begin{itemize}
		\item $ \ket{\Psi_{0}} = \ket{\psi_S} \otimes \ket{\phi_a} = \ket{\psi_S} \otimes  \ket{0_a} $ \\
		\item $ \ket{\Psi_{1}} = \ket{\psi_S} \otimes \ket{\phi_a} = \ket{\psi_S} \otimes  \ket{1_a} $
	\end{itemize} 
	
	Let's first focus on the $ \ket{\Psi_{0}} $ initial state:
	
	\begin{align}
		\ket{\Psi'_{0}} = \mathcal{U} \ket{\Psi_{0}}
		&= \cos{\left(c_j \Delta t\right)} \ket{\psi_{S}} \otimes \ket{0_a} - i\sin{\left(c_j \Delta t \right)} \sigma_{z}^{j} \ket{\psi_{S}} \otimes \sigma_{z}^{a_j}\ket{0_a} \notag \\
		&= \cos{\left(c_j \Delta t\right)} \ket{\psi_{S}} \otimes \ket{0_a} - i\sin{\left(c_j \Delta t \right)} \sigma_{z}^{j} \ket{\psi_{S}} \otimes \ket{0_a} \notag \\
		&= \left[\cos{\left(c_j \Delta t\right)} - i\sin{\left(c_j \Delta t \right)} \sigma_{z}^{j} \right] \ket{\psi_{S}} \otimes \ket{0_a}
	\end{align}
	
	And then on the $ \ket{\Psi_{1}} $ initial state:
	
	\begin{align}
		\ket{\Psi'_{1}} = \mathcal{U} \ket{\Psi_{1}}
		&= \cos{\left(c_j \Delta t\right)} \ket{\psi_{S}} \otimes \ket{1_a} - i\sin{\left(c_j \Delta t \right)} \sigma_{z}^{j} \ket{\psi_{S}} \otimes \sigma_{z}^{a_j}\ket{1_a} \notag \\
		&= \cos{\left(c_j \Delta t\right)} \ket{\psi_{S}} \otimes \ket{1_a} + i\sin{\left(c_j \Delta t \right)} \sigma_{z}^{j} \ket{\psi_{S}} \otimes \ket{1_a} \notag \\
		&= \left[\cos{\left(c_j \Delta t\right)} + i\sin{\left(c_j \Delta t \right)} \sigma_{z}^{j} \right] \ket{\psi_{S}} \otimes \ket{1_a}
	\end{align}
	
	\subsection{Density Matrix Reconstruction}
	
	We can now define the effective operators acting on the system space. Let us define the operator $\mathcal{K}_0$ acting on $\ket{\psi_S}$ when the ancilla is $\ket{0}$:
	\begin{equation}
		\mathcal{K}_0 = \cos{\left( c_j \Delta t \right)} - i \sin{\left( c_j \Delta t \right)} \, \sigma_{z}^{j} = e^{-i c_j \Delta t \sigma_z^j}
	\end{equation}
	
	And the operator $\mathcal{K}_1$ when the ancilla is $\ket{1}$:
	\begin{equation}
		\mathcal{K}_1 = \cos{\left( c_j \Delta t \right)} + i \sin{\left( c_j \Delta t \right)} \, \sigma_{z}^{j} = e^{+i c_j \Delta t \sigma_z^j} = \mathcal{K}_0^\dagger
	\end{equation}
	
	Note that $\mathcal{K}_1$ is simply the Hermitian conjugate (and inverse) of $\mathcal{K}_0$. \\
	
	\begin{flushleft}
		Let's now calculate the evolution of the completely mixed Density Matrix, reconstructing it's term from the wave function defined above:
	\end{flushleft}
	
	\begin{align}
		\rho'_{0} &= \ket{\Psi'_{0}}\bra{\Psi'_{0}} = \mathcal{K}_{0} \ket{\psi_S}\bra{\psi_S} \mathcal{K}_{0}^{\dagger} \otimes \ket{0_a}\bra{0_a} = \mathcal{K}_{0} \ket{\psi_S}\bra{\psi_S} \mathcal{K}_{1} \otimes \ket{0_a}\bra{0_a} \notag \\ 
		& = \Big [ \left(\cos{\left( c_j \Delta t \right)} - i \sin{\left( c_j \Delta t \right)} \, \sigma_{z}^{j}\right) \, \ket{\psi_S}\bra{\psi_S} \, \left(\cos{\left( c_j \Delta t \right)} + i \sin{\left( c_j \Delta t \right)} \, \sigma_{z}^{j}\right) \Big]    \otimes \ket{0_a}\bra{0_a} \notag \\
		&= \Big [ \cos^{2}{\left(c_{j} \Delta t\right)} \ket{\psi_S}\bra{\psi_S} + \sin^{2}{\left(c_{j} \Delta t\right)} \, \sigma_{z} \ket{\psi_S}\bra{\psi_S} \sigma_{z} \notag  \\
		& \quad + i \sin{\left(c_{j} \Delta t \right)} \cos{\left(c_{j} \Delta t\right)} \,  \ket{\psi_S}\bra{\psi_S} \sigma_{z} -i \sin{\left(c_{j} \Delta t \right)} \cos{\left(c_{j} \Delta t\right)} \, \sigma_{z} \ket{\psi_S}\bra{\psi_S} \Big] \otimes \ket{0_a}\bra{0_a} \notag \\
		&= \Big [ \cos^{2}{\left(c_{j} \Delta t\right)} \rho_S + \sin^{2}{\left(c_{j} \Delta t\right)} \, \sigma_{z} \rho_S \sigma_{z} \notag \\
		& \quad + i \sin{\left(c_{j} \Delta t \right)} \cos{\left(c_{j} \Delta t\right)} \, \rho_S \sigma_{z} - i \sin{\left(c_{j} \Delta t \right)} \cos{\left(c_{j} \Delta t\right)} \, \sigma_{z} \rho_S \Big] \otimes \ket{0_a}\bra{0_a} \\
		\notag \\
		\rho'_{1} &= \ket{\Psi'_{1}}\bra{\Psi'_{1}} = \mathcal{K}_{1} \ket{\psi_S}\bra{\psi_S} \mathcal{K}_{1}^{\dagger} \otimes \ket{1_a}\bra{1_a} = \mathcal{K}_{1} \ket{\psi_S}\bra{\psi_S} \mathcal{K}_{0} \otimes \ket{1_a}\bra{1_a} \notag \\ 
		& = \Big [ \left(\cos{\left( c_j \Delta t \right)} + i \sin{\left( c_j \Delta t \right)} \, \sigma_{z}^{j}\right) \, \ket{\psi_S}\bra{\psi_S} \, \left(\cos{\left( c_j \Delta t \right)} - i \sin{\left( c_j \Delta t \right)} \, \sigma_{z}^{j}\right) \Big]    \otimes \ket{1_a}\bra{1_a} \notag \\
		&= \Big [ \cos^{2}{\left(c_{j} \Delta t\right)} \ket{\psi_S}\bra{\psi_S} + \sin^{2}{\left(c_{j} \Delta t\right)} \, \sigma_{z} \ket{\psi_S}\bra{\psi_S} \sigma_{z} \notag \\
		& \quad - i \sin{\left(c_{j} \Delta t \right)} \cos{\left(c_{j} \Delta t\right)} \,  \ket{\psi_S}\bra{\psi_S} \sigma_{z} + i \sin{\left(c_{j} \Delta t \right)} \cos{\left(c_{j} \Delta t\right)} \, \sigma_{z} \ket{\psi_S}\bra{\psi_S}  \Big] \otimes \ket{1_a}\bra{1_a} \notag \\
		&= \Big [ \cos^{2}{\left(c_{j} \Delta t\right)} \rho_S + \sin^{2}{\left(c_{j} \Delta t\right)} \, \sigma_{z} \rho_S \sigma_{z} \notag \\
		& \quad - i \sin{\left(c_{j} \Delta t \right)} \cos{\left(c_{j} \Delta t\right)} \, \rho_S \sigma_{z} + i \sin{\left(c_{j} \Delta t \right)} \cos{\left(c_{j} \Delta t\right)} \, \sigma_{z} \rho_S \Big] \otimes \ket{1_a}\bra{1_a}
	\end{align}
	
	The complete Density Matrix can be reconstructed as : 
	\begin{equation}
		\rho' = \frac{1}{2} \rho'_{0} + \frac{1}{2} \rho'_{1} 
	\end{equation} 
	
	Note that the coherent oscillation, i.e. the cross terms $ i \sin{\left(c_{j} \Delta t\right)} \, \cos{\left(c_{j} \Delta t\right)} $ cancel each other, so that we obtain :
	
	\begin{align}
		\rho' &= \frac{1}{2} \Big[ \cos^{2}{\left(c_{j} \Delta t\right)} \rho_S + \sin^{2}{\left(c_{j} \Delta t\right)} \, \sigma_{z} \rho_S \sigma_{z} \Big] \otimes \ket{0_a}\bra{0_a} \notag \\
		& + \frac{1}{2} \Big[ \cos^{2}{\left(c_{j} \Delta t\right)} \rho_S + \sin^{2}{\left(c_{j} \Delta t\right)} \, \sigma_{z} \rho_S \sigma_{z} \Big] \otimes \ket{1_a}\bra{1_a}
	\end{align}
	
	\begin{flushleft}
		If we now take the partial Trace over our total density matrix we obtain the System Density Matrix:
	\end{flushleft}
	
	\begin{align}
		\rho'_S & = Tr_a(\rho') = \sum_{k=1}^{N_{anc}}{\bra{\phi_k}\rho'\ket{\phi_k}} \notag \\
		&= \cos^{2}{\left(c_{j} \Delta t\right)} \rho_S + \sin^{2}{\left(c_{j} \Delta t\right)} \, \sigma_{z} \rho_S \sigma_{z} 
	\end{align}	
	
	\begin{flushleft}
		It can be demonstrated that this equation can lead to a Dynamical Map in the Lindblad Master Equation form.  
	\end{flushleft}
	
	\subsection{Recovery of Lindblad form}
	
	Since Eq.35 is the same obtained for the Quantum Jump limit, the procedure to recover the Lindblad form is the same of section 2.1.2
	
	\subsection{Algorithm for Trajectories}
	
	As already seen in Section 2.1.3 to produce an evolution with the Collisional Hamiltonian we can evolve the density matrix in tensor product with the Ancilla's density matrix with the complete Hamiltonian, i.e. Excitonic and Collisional term, and then trace out the Ancilla; this evolution has to reproduce exactly the Lindblad dynamic and corresponds to an average over an infinite number of trajectories. \\
	The latter ones are the second method to evolve our system via single wave function dynamics; a trajectories, as seen so far, are stochastic evolution of the system wf, derived from the physical effect of measuring the Ancilla's state after every collision. An algorithm to reproduce this effect can be obtained with a sort of Monte Carlo method in which we extract a random number and if a certain condition is met we apply a particular evolution effect to the wave function. In the Quantum Jump limit the condition was the probability associated to the measurement of the Ancilla's state $ \ket{1_a} $, explicitly derived. In the Diffusive limit we cannot base the condition on the norm of the evolved state, but we have to define a priori the its value, in order to reproduce the effect described by the Ancilla's density matrix. Since the latter is a completely mixed density matrix, i.e. both $ \ket{0_a} $ and $ \ket{1_a} $ state of the Ancilla have probability 1/2 to exist (it's like having a finite number of Ancillas where half of them are in the forund state and the other are in the excited state). So in this case the algorithm will be:
	
	\begin{itemize}
		\item [1.] Apply the unitary evolution associate to $ \mathcal{H}_{site} $ and the hopping \\ potential V for a discrete time step $ \Delta t $ 
		\item [2.] Extract a random number between 0 and 1
		\item [3.] If it's less than 0.5 we'll follow Eq.27, otherwise Eq.28
		\item [4.] Record the site population at that time, then restart the algorithm
	\end{itemize} 
	
	By averaging between multiple random trajectories, it is possible to reconstruct the average dynamics that reproduce the Lindblad evolution.
	
	\clearpage
	\newpage


\end{document}
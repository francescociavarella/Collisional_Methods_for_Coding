\documentclass[a4paper,12pt]{article}

\usepackage[utf8]{inputenc}
\usepackage[T1]{fontenc}
\usepackage{titlesec}
\usepackage{amsmath}
\usepackage{amssymb}
\usepackage{hyperref}
\usepackage{braket}
\usepackage{tikz}
\usepackage{cancel}
\usepackage[top=3cm, bottom=3cm, left=2cm, right=2cm]{geometry}
\usepackage{graphicx}
\usepackage{float}
\usetikzlibrary{arrows.meta, angles, quotes}

\allowdisplaybreaks

\titleformat{\section}
{\normalfont\normalsize\bfseries}
{\thesection}{1em}{}
\titleformat{\subsection}
{\normalfont\small\bfseries}
{\thesubsection}{1em}{}
\titleformat{\subsubsection}
{\normalfont\small\bfseries}
{\thesubsubsection}{1em}{}

\title{
	\vspace{2cm}
	\Huge \textbf{Demontrations} \\[0.5cm]
	\large Collection of notes, formulas and demonstrations in the field of dynamics of open quantum systems and Collisional Methods
	\vspace{5cm}
}

\author{\Large Francesco Ciavarella - University of Padua \\[10pt] \href{mailto:francesco.ciavarella@studenti.unipd.it}{francesco.ciavarella@studenti.unipd.it}}

\begin{document}
	
	\maketitle
	\thispagestyle{empty} 
	\newpage
	
	\tableofcontents
	\newpage
	
	\section{Quantum Jump limit in Collisional Methods}
	\label{sec:Quantum_Jump_limit_in_Collisional_Methods}
	\subsection{Evolution Operator and Ancilla Initialization}
	\label{sub:Evolution_Operator_and_Ancilla_Initialization_QJ}
	The goal is to reproduce the Lindbald dynamic using a Collsional Model, that gives us a trajectory of evolution of a single state of the density matrix $ \ket{\Psi_k} $ rappresentig a qubit. By repeting the dynamic several times it's possibile to recrate the Lindblad evolution of the density matrix.
	Different unravelling of the $ \rho (t) $ can be achieved by using different Collsional Hamiltonian. First we will reproduce the so called "Quantum Jump" or "Monte Carlo" limit, in which essentially the ancilla states stochastically jumps between the states $ \ket{0_a} $ and $ \ket{1_a} $ with a small probability to move to $ \ket{1} $ related to an effective collision, wich corresponds to an application of a $ \sigma_{z} $ to the system's state. \\
	The associated Hamiltonian acting on the System-Ancilla states is:
	\begin{gather}
		\mathcal{H}_{CM} = \mathcal{H}_{exc} + \mathcal{H}_{collision} \\
		= \sum_{j=1}^{N} {\frac{\varepsilon_j}{2} \sigma_{z}^{j} \otimes \mathbb{I}^{\otimes N} } + \sum_{\langle j,j' \rangle} {\frac{V_{j,j'}}{2} \left( \sigma_{x}^{j}\sigma_{x}^{j'} \otimes \mathbb{I}^{\otimes N} + \sigma_{y}^{j}\sigma_{y}^{j'} \otimes \mathbb{I}^{\otimes N} \right)} + \sum_{j=1}^{N}{c_j \sigma_{z}^{j} \otimes \sigma_{x}^{a_j}}
		\label{eq:CM_hamiltonian}
	\end{gather}
	where $ \sigma_{\alpha}^j = \mathbb{I}^{\otimes j-1} \otimes \sigma_{\alpha} \otimes \mathbb{I}^{\otimes N-j}  $ . \\
	The collision force $ c_j $ is related to the Lindblad phase shift constant by the equation $ c_j = \sqrt{\gamma_j / 4 \Delta t} $. 
	Let's focus on the Interaction Hamiltonian and how it acts on an already enatgled system-ancilla states $ \ket{\Psi} = \ket{\Psi_S} \otimes \ket{0_a} $ \\
	\\
	The evolution operator based on $ \mathcal{H}_{collsion} $ is :
	\begin{equation}
		U_{collsion} = \exp \left(-i c_j \sigma_{z}^{j} \otimes \sigma_{x}^{a_j} \Delta t\right) = \cos \left(c_j \Delta t \right) \mathbb{I}^j \otimes \mathbb{I}^a - i \sin \left(c_j \Delta t \right) \sigma_{z}^{j} \otimes \sigma_{x}^{a_j}
		\label{eq:U_QJ}
	\end{equation}
	And applied to $ \ket{\Psi} $ gives:
	\begin{align} 
		\ket{\Psi'} &= \cos \left(c_j \Delta t \right) \mathbb{I}^j \otimes \mathbb{I}^a \ket{\Psi} - i \sin \left(c_j \Delta t \right) \sigma_{z}^{j} \otimes \sigma_{x}^{a} \ket{\Psi} \notag \\
		&= \cos \left(c_j \Delta t \right) \ket{\Psi_S} \otimes \ket{0_a} - i \sin \left(c_j \Delta t \right) \sigma_{z}^{j} \ket{\Psi_S} \otimes \sigma_{x}^{a} \ket{0_a} \notag \\
		&= \cos \left(c_j \Delta t \right) \ket{\Psi_S} \otimes \ket{0_a} - i \sin \left(c_j \Delta t \right) \sigma_{z}^{j} \ket{\Psi_S} \otimes \ket{1_a}
		\label{eq:psi_evo_QJ}
	\end{align}
	
	\subsection{Reconstruction of the Density Matrix}
	\label{sub:Reconstruction_of_the_Density_Matrix_QJ}
	Now we can rebuild the Density Matrix as :
	\begin{align}
		\rho' &= \ket{\Psi'}\bra{\Psi'} = \cos^{2} \left(c_j \Delta t \right) \rho_S' \otimes \ket{0_a}\bra{0_a} + \sin^{2} \left(c_j \Delta t \right) \sigma_{z}^{j} \rho_S \sigma_{z}^{j} \otimes \ket{1_a}\bra{1_a} \notag \\
		&+ \cos \left(c_j \Delta t \right) \sin \left(c_j \Delta t \right) \rho_S \sigma_{z}^{j} \otimes \ket{0_a} \bra{1_a} + \cos \left(c_j \Delta t \right) \sin \left(c_j \Delta t \right) \sigma_{z}^{j} \rho_S \otimes \ket{1_a} \bra{0_a}
		\label{eq:rho_reconstruction}
	\end{align}
	
	\begin{flushleft}
		If we now take the partial Trace over our total density matrix we obtain the System Density Matrix:
	\end{flushleft}
	
	\begin{align}
		\rho'_S & = Tr_a(\rho') = \sum_{k=1}^{N_{anc}}{\bra{\phi_k}\rho'\ket{\phi_k}} \notag \\
		&= \cos^{2}{\left(c_{j} \Delta t\right)} \rho_S + \sin^{2}{\left(c_{j} \Delta t\right)} \, \sigma_{z} \rho_S \sigma_{z} 
		\label{eq:rho_S_anc_trace}
	\end{align}	
	
	\subsection{Recovery of Lindblad form}
	\label{sub:Recovery_of_Lindblad_form_QJ}
	Before working on Eq.\eqref{eq:rho_S_anc_trace} we have to point out that so far we have only analyzed the effect of the Collisional Hamiltonian, forgetting about the unitary evolution via the site Hamiltonian, which has the form : 
	
	\begin{equation}
		\mathcal{H}_{exc} = \sum_{j=1}^{N} {\frac{\varepsilon_j}{2} \sigma_{z}^{j} \otimes \mathbb{I}^{\otimes N} } + \sum_{\langle j,j' \rangle} {\frac{V_{j,j'}}{2} \left( \sigma_{x}^{j}\sigma_{x}^{j'} \otimes \mathbb{I}^{\otimes N} + \sigma_{y}^{j}\sigma_{y}^{j'} \otimes \mathbb{I}^{\otimes N} \right)}
		\label{eq:H_exc}
	\end{equation}
	
	In this case we already know that applying the Time Evolution Operator $ \mathcal{U}_{exc} = \exp {\left( -i \mathcal{H}_{exc}\Delta t \right)} $ in the first order in $ \Delta t $ approximation, we obtain the Liouville-von Neumann Equation, which can be written as : 
	
	\begin{align}
		\rho_S \left(t + \Delta t\right) = \rho_S (t) - i \left[\mathcal{H}_{exc}, \rho_S \right] \Delta t \\
		\frac{\rho_S \left(t + \Delta t\right) - \rho_S (t)}{\Delta t} = - i \left[\mathcal{H}_{exc}, \rho_S \right]
		\label{eq:Liouville_von_Neumann}
	\end{align} 
	
	Let's focus now on Eq.\eqref{eq:rho_S_anc_trace} and its form for $ \lim \Delta t \rightarrow 0  $; using Taylor's expansion for the $ sin $ and $ cos $ we obtain that:
	
	\begin{itemize}
		\item $ \cos^{2}{\left(c_{j} \Delta t \right)} \approx 1 - c_{j}^{2} \Delta t ^{2} $
		\item $ \sin^{2}{\left(c_{j} \Delta t \right)} \approx c_{j}^{2} \Delta t ^{2} $
		\label{eq:sin_cos_approx}
	\end{itemize} 
	\begin{align}
		\rho_S \left(t + \Delta t\right) & = \cos^{2}{\left(c_{j} \Delta t\right)} \rho_S (t) + \sin^{2}{\left(c_{j} \Delta t\right)} \, \sigma_{z} \rho_S (t) \sigma_{z} \notag \\ 
		& \approx \rho_S (t) + c_{j}^{2} \Delta t ^{2} \left( \sigma_{z} \rho_S \sigma_{z} - \rho_S (t) \right)
		\label{eq:rho_approx}
	\end{align}
		
	The total dynamics over a time step $\Delta t$ can be obtained by composing the unitary evolution generated by $\mathcal{H}_{exc}$ and the collisional map derived above. Using the "Trotter decomposition" to first order in $\Delta t$, we approximate the total evolution operator as the product of the individual propagators:
	
	\begin{equation}
		\mathcal{U}_{tot}(\Delta t) \approx \mathcal{U}_{exc}(\Delta t) \cdot \mathcal{U}_{coll}(\Delta t)
		\label{eq:trotter_suzuki}
	\end{equation}
	
	Applying this composition sequentially to the density matrix (i.e., applying the Hamiltonian evolution to the state resulting from the collision), we obtain:
	
	\begin{align}
		&\rho_S \left(t + \Delta t\right) = \rho_S (t) - i \left[\mathcal{H}_{exc}, \rho_S \right]\Delta t + \sum_{j=1}^{N}{c_{j}^{2} \Delta t ^{2} \left( \sigma_{z}^{j} \rho_S \sigma_{z}^{j} - \rho_S (t) \right)} \\
		&\frac{\rho_S \left(t + \Delta t\right) - \rho_S (t)}{\Delta t} = - i \left[\mathcal{H}_{exc}, \rho_S \right] + \sum_{j=1}^{N}{c_{j}^{2} \Delta t \left( \sigma_{z}^{j} \rho_S \sigma_{z}^{j} - \rho_S (t) \right)}
		\label{eq_incremental_ratio} 
	\end{align}
	
	We have to set $ c_{j}^{2} = \Gamma_j / \Delta t  $ to recover the differential form : 
	
	\begin{equation}
		\lim\limits_{\Delta t \rightarrow 0} \quad \dot{\rho_S} = - i \left[\mathcal{H}_{exc}, \rho_S \right] + \sum_{j=1}^{N}{\Gamma_{j} \left( \sigma_{z}^{j} \rho_S \sigma_{z}^{j} - \rho_S (t) \right)}
		\label{eq:corrisponding_ME}
	\end{equation}
	
	Eq.\eqref{eq:corrisponding_ME} corresponds to the Lindblad ME setting $ \mathcal{L}_k = \sigma_{z} $ in fact $ \sigma_{z} = \sigma_{z}^{\dagger} $ and $ \sigma_{z} \sigma_{z}^{\dagger} = \mathbb{I} $	
	
	\clearpage
	\newpage
		
	\section{Diffusive Limit in Collisional Methods}
	\label{sec:Diffusive_Limit_in_Collisional_Methods}
	
	\subsection{Evolution Operator and Ancilla Initialization}
	\label{sub:Evolution_Operator_and_Ancilla_Initialization_Diff}
	
	In this framework, we define the interaction Hamiltonian and the initial state of the ancilla as follows:
	\begin{align} 
		&\mathcal{H}_{CM} = c_{j} \sigma_{z}^{j} \otimes \sigma_{z}^{a_j} \quad \text{and} \quad \rho_{a} = \frac{1}{2} \, \mathbb{I}^{a_j} =\rho_{a} = \frac{1}{2} \begin{pmatrix} 1 & 0 \\ 0 & 1 \end{pmatrix} 
		\label{eq:Diff_initialization}
	\end{align}
	
	The definition of the initial state of the Ancilla as a completely mixed state is crucial. It implies that the wave function for the trajectory development will be initialized with probability $p=1/2$ in the state $ \ket{\phi_a} = \ket{0_a} $ and with probability $p=1/2$ in the state $ \ket{\phi_a} = \ket{1_a} $. \\
	
	The Unitary Evolution Operator reads: 
	\begin{equation}
		\mathcal{U} = \exp{\left( -i c_j \Delta t \, \sigma_{z}^{j} \otimes \sigma_{z}^{a_j} \right)}
		\label{eq:U_Diff}
	\end{equation}
	
	First we expand in the Taylor's Series the exponential which reads:
	\begin{equation}
		\exp{\left( -i c_j \Delta t \, \sigma_{z}^{j} \otimes \sigma_{z}^{a_j} \right)} = \, \sum_{n=0}^{\infty}{\frac{\left( -i c_j \Delta t \, \sigma_{z}^{j} \otimes \sigma_{z}^{a_j} \right)^n}{n!} } \\
		\label{eq:exp_decomposition}
	\end{equation}
	
	\begin{align*}
		&= 1 - i c_j \Delta t \, \sigma_{z}^{j} \otimes \sigma_{z}^{a_j} - \frac{\left( c_j \Delta t \, \sigma_{z}^{j} \otimes \sigma_{z}^{a_j} \right)^2}{2!} + i \frac{\left( c_j \Delta t \, \sigma_{z}^{j} \otimes \sigma_{z}^{a_j} \right)^3}{3!} + \frac{\left( c_j \Delta t \, \sigma_{z}^{j} \otimes \sigma_{z}^{a_j} \right)^4}{4!} + \cdots  \\
		&=1 - \frac{\left( c_j \Delta t \, \sigma_{z}^{j} \otimes \sigma_{z}^{a_j} \right)^2}{2!} + \cdots  + \frac{\left( -i c_j \Delta t \, \sigma_{z}^{j} \otimes \sigma_{z}^{a_j} \right)^{{2n}}}{2n!}  \\
		&\,- i c_j\Delta t \, \sigma_{z}^{j} \otimes \sigma_{z}^{a_j} + \cdots  + \frac{\left( -i c_j \Delta t \, \sigma_{z}^{j} \otimes \sigma_{z}^{a_j} \right)^{{2n+1}}}{(2n+1)!} \\ 
	\end{align*}
	
	\begin{flushleft}
		We now apply th Pauli Matrices property to calculate the exponential term: 
	\end{flushleft}
	
	\begin{gather}
		\left(\sigma_{z}^{j} \otimes \sigma_{z}^{a_j} \right)^{2} = \sigma_{z}^{j} \sigma_{z}^{j} \otimes \sigma_{z}^{a_j} \sigma_{z}^{a_j} = \mathbb{I}^{j} \otimes \mathbb{I}^{a_j} \\
		\left(\sigma_{z}^{j} \otimes \sigma_{z}^{a_j} \right)^{3} = \sigma_{z}^{j} \otimes \sigma_{z}^{a_j}
		\label{eq:pauli_matrix_prop}
	\end{gather}
		
	\begin{flushleft}
		So we can rewrite the summation and make evident the \textit{cos} and \textit{sin} expansion:
	\end{flushleft}
	
	\begin{align}
		\exp{\left( -i c_j \Delta t \, \sigma_{z}^{j} \otimes \sigma_{z}^{a_j} \right)} &=\mathbb{I}^{j} \otimes \mathbb{I}^{a_j} \, \sum_{2n=0}^{\infty}
		{ \frac{\left( c_j \Delta t \right)^{{2n}}}{2n!} } \,  - i \left(\sigma_{z}^{j} \otimes \sigma_{z}^{a_j} \right) \sum_{2n=0}^{\infty}{ \frac{\left( c_j \Delta t  \right)^{{2n+1}}}{(2n+1)!} } \notag \\
		\notag \\	
		&= \cos{\left(c_j \Delta t\right)} \mathbb{I}^{j} \otimes \mathbb{I}^{a_j} \, - \sin{\left(c_j \Delta t \right)} \sigma_{z}^{j} \otimes \sigma_{z}^{a_j}
		\label{eq:U_diff_reconstruction}
	\end{align}
	
	\newpage
	
	\begin{flushleft}
		Now we can apply the evolution operator to the two different initial state which are :
	\end{flushleft}
	
	\begin{itemize}
		\item $ \ket{\Psi_{0}} = \ket{\psi_S} \otimes \ket{\phi_a} = \ket{\psi_S} \otimes  \ket{0_a} $ \\
		\item $ \ket{\Psi_{1}} = \ket{\psi_S} \otimes \ket{\phi_a} = \ket{\psi_S} \otimes  \ket{1_a} $
		\label{eq:psi_iniz_0_1}
	\end{itemize} 
	
	Let's first focus on the $ \ket{\Psi_{0}} $ initial state:
	\begin{align}
		\ket{\Psi'_{0}} = \mathcal{U} \ket{\Psi_{0}}
		&= \cos{\left(c_j \Delta t\right)} \ket{\psi_{S}} \otimes \ket{0_a} - i\sin{\left(c_j \Delta t \right)} \sigma_{z}^{j} \ket{\psi_{S}} \otimes \sigma_{z}^{a_j}\ket{0_a} \notag \\
		&= \cos{\left(c_j \Delta t\right)} \ket{\psi_{S}} \otimes \ket{0_a} - i\sin{\left(c_j \Delta t \right)} \sigma_{z}^{j} \ket{\psi_{S}} \otimes \ket{0_a} \notag \\
		&= \left[\cos{\left(c_j \Delta t\right)} - i\sin{\left(c_j \Delta t \right)} \sigma_{z}^{j} \right] \ket{\psi_{S}} \otimes \ket{0_a}
		\label{eq:psi_0_evo}
	\end{align}
	
	And then on the $ \ket{\Psi_{1}} $ initial state:
	\begin{align}
		\ket{\Psi'_{1}} = \mathcal{U} \ket{\Psi_{1}}
		&= \cos{\left(c_j \Delta t\right)} \ket{\psi_{S}} \otimes \ket{1_a} - i\sin{\left(c_j \Delta t \right)} \sigma_{z}^{j} \ket{\psi_{S}} \otimes \sigma_{z}^{a_j}\ket{1_a} \notag \\
		&= \cos{\left(c_j \Delta t\right)} \ket{\psi_{S}} \otimes \ket{1_a} + i\sin{\left(c_j \Delta t \right)} \sigma_{z}^{j} \ket{\psi_{S}} \otimes \ket{1_a} \notag \\
		&= \left[\cos{\left(c_j \Delta t\right)} + i\sin{\left(c_j \Delta t \right)} \sigma_{z}^{j} \right] \ket{\psi_{S}} \otimes \ket{1_a}
		\label{eq:psi_1_evo}
	\end{align}
	
	\subsection{Density Matrix Reconstruction}
	\label{sub:Density_Matrix_Reconstruction_Diff}
	
	We can now define the effective operators acting on the system space. Let us define the operator $\mathcal{K}_0$ acting on $\ket{\psi_S}$ when the ancilla is $\ket{0}$:
	\begin{equation}
		\mathcal{K}_0 = \cos{\left( c_j \Delta t \right)} - i \sin{\left( c_j \Delta t \right)} \, \sigma_{z}^{j} = e^{-i c_j \Delta t \sigma_z^j}
		\label{eq:k0_def}
	\end{equation}
	
	And the operator $\mathcal{K}_1$ when the ancilla is $\ket{1}$:
	\begin{equation}
		\mathcal{K}_1 = \cos{\left( c_j \Delta t \right)} + i \sin{\left( c_j \Delta t \right)} \, \sigma_{z}^{j} = e^{+i c_j \Delta t \sigma_z^j} = \mathcal{K}_0^\dagger
		\label{eq:k1_def}
	\end{equation}
	
	Note that $\mathcal{K}_1$ is simply the Hermitian conjugate (and inverse) of $\mathcal{K}_0$.
	
	\begin{flushleft}
		Let's now calculate the evolution of the completely mixed Density Matrix, reconstructing it's term from the wave function defined above:
	\end{flushleft}
	\begin{align}
		\rho'_{0} &= \ket{\Psi'_{0}}\bra{\Psi'_{0}} = \mathcal{K}_{0} \ket{\psi_S}\bra{\psi_S} \mathcal{K}_{0}^{\dagger} \otimes \ket{0_a}\bra{0_a} = \mathcal{K}_{0} \ket{\psi_S}\bra{\psi_S} \mathcal{K}_{1} \otimes \ket{0_a}\bra{0_a} \notag \\ 
		& = \Big [ \left(\cos{\left( c_j \Delta t \right)} - i \sin{\left( c_j \Delta t \right)} \, \sigma_{z}^{j}\right) \, \ket{\psi_S}\bra{\psi_S} \, \left(\cos{\left( c_j \Delta t \right)} + i \sin{\left( c_j \Delta t \right)} \, \sigma_{z}^{j}\right) \Big]    \otimes \ket{0_a}\bra{0_a} \notag \\
		&= \Big [ \cos^{2}{\left(c_{j} \Delta t\right)} \ket{\psi_S}\bra{\psi_S} + \sin^{2}{\left(c_{j} \Delta t\right)} \, \sigma_{z} \ket{\psi_S}\bra{\psi_S} \sigma_{z} \notag  \\
		& \quad + i \sin{\left(c_{j} \Delta t \right)} \cos{\left(c_{j} \Delta t\right)} \,  \ket{\psi_S}\bra{\psi_S} \sigma_{z} -i \sin{\left(c_{j} \Delta t \right)} \cos{\left(c_{j} \Delta t\right)} \, \sigma_{z} \ket{\psi_S}\bra{\psi_S} \Big] \otimes \ket{0_a}\bra{0_a} \notag \\
		&= \Big [ \cos^{2}{\left(c_{j} \Delta t\right)} \rho_S + \sin^{2}{\left(c_{j} \Delta t\right)} \, \sigma_{z} \rho_S \sigma_{z} \notag \\
		& \quad + i \sin{\left(c_{j} \Delta t \right)} \cos{\left(c_{j} \Delta t\right)} \, \rho_S \sigma_{z} - i \sin{\left(c_{j} \Delta t \right)} \cos{\left(c_{j} \Delta t\right)} \, \sigma_{z} \rho_S \Big] \otimes \ket{0_a}\bra{0_a} \\
		\notag \\
		\notag \\
		\rho'_{1} &= \ket{\Psi'_{1}}\bra{\Psi'_{1}} = \mathcal{K}_{1} \ket{\psi_S}\bra{\psi_S} \mathcal{K}_{1}^{\dagger} \otimes \ket{1_a}\bra{1_a} = \mathcal{K}_{1} \ket{\psi_S}\bra{\psi_S} \mathcal{K}_{0} \otimes \ket{1_a}\bra{1_a} \notag \\ 
		& = \Big [ \left(\cos{\left( c_j \Delta t \right)} + i \sin{\left( c_j \Delta t \right)} \, \sigma_{z}^{j}\right) \, \ket{\psi_S}\bra{\psi_S} \, \left(\cos{\left( c_j \Delta t \right)} - i \sin{\left( c_j \Delta t \right)} \, \sigma_{z}^{j}\right) \Big]    \otimes \ket{1_a}\bra{1_a} \notag \\
		&= \Big [ \cos^{2}{\left(c_{j} \Delta t\right)} \ket{\psi_S}\bra{\psi_S} + \sin^{2}{\left(c_{j} \Delta t\right)} \, \sigma_{z} \ket{\psi_S}\bra{\psi_S} \sigma_{z} \notag \\
		& \quad - i \sin{\left(c_{j} \Delta t \right)} \cos{\left(c_{j} \Delta t\right)} \,  \ket{\psi_S}\bra{\psi_S} \sigma_{z} + i \sin{\left(c_{j} \Delta t \right)} \cos{\left(c_{j} \Delta t\right)} \, \sigma_{z} \ket{\psi_S}\bra{\psi_S}  \Big] \otimes \ket{1_a}\bra{1_a} \notag \\
		&= \Big [ \cos^{2}{\left(c_{j} \Delta t\right)} \rho_S + \sin^{2}{\left(c_{j} \Delta t\right)} \, \sigma_{z} \rho_S \sigma_{z} \notag \\
		& \quad - i \sin{\left(c_{j} \Delta t \right)} \cos{\left(c_{j} \Delta t\right)} \, \rho_S \sigma_{z} + i \sin{\left(c_{j} \Delta t \right)} \cos{\left(c_{j} \Delta t\right)} \, \sigma_{z} \rho_S \Big] \otimes \ket{1_a}\bra{1_a}
		\label{eq:rho_0_1_reconstruction}
	\end{align}
	
	The complete Density Matrix can be reconstructed as : 
	\begin{equation}
		\rho' = \frac{1}{2} \rho'_{0} + \frac{1}{2} \rho'_{1} 
		\label{eq:rho_complete_reconstruction}
	\end{equation} 
	
	Note that the coherent oscillation, i.e. the cross terms $ i \sin{\left(c_{j} \Delta t\right)} \, \cos{\left(c_{j} \Delta t\right)} $ cancel each other, so that we obtain :
	
	\begin{align}
		\rho' &= \frac{1}{2} \Big[ \cos^{2}{\left(c_{j} \Delta t\right)} \rho_S + \sin^{2}{\left(c_{j} \Delta t\right)} \, \sigma_{z} \rho_S \sigma_{z} \Big] \otimes \ket{0_a}\bra{0_a} \notag \\
		& + \frac{1}{2} \Big[ \cos^{2}{\left(c_{j} \Delta t\right)} \rho_S + \sin^{2}{\left(c_{j} \Delta t\right)} \, \sigma_{z} \rho_S \sigma_{z} \Big] \otimes \ket{1_a}\bra{1_a}
		\label{eq:rho_complete_explicit}
	\end{align}

	If we now take the partial Trace over our total density matrix we obtain the System Density Matrix:	
	\begin{align}
		\rho'_S & = Tr_a(\rho') = \sum_{k=1}^{N_{anc}}{\bra{\phi_k}\rho'\ket{\phi_k}} \notag \\
		&= \cos^{2}{\left(c_{j} \Delta t\right)} \rho_S + \sin^{2}{\left(c_{j} \Delta t\right)} \, \sigma_{z} \rho_S \sigma_{z} 
		\label{eq:rho_s_partial_trace_diff}
	\end{align}	
	
	\begin{flushleft}
		It can be demonstrated that this equation can lead to a Dynamical Map in the Lindblad Master Equation form.  
	\end{flushleft}
	
	\subsection{Recovery of Lindblad form}
	\label{sub:Recovery_of_Lindblad_form_Diff}
	
	Since Eq.\eqref{eq:rho_s_partial_trace_diff} is the same obtained for the Quantum Jump limit, the procedure to recover the Lindblad form is the same of Section \ref{sub:Recovery_of_Lindblad_form_QJ}

\end{document}